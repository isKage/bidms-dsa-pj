\documentclass{beamer}
\usepackage{amsfonts,amsmath,oldgerm}
\usetheme{sintef}
\usepackage{xeCJK}
\usepackage{ctex}
\usepackage{hyperref}

\newcommand{\testcolor}[1]{\colorbox{#1}{\textcolor{#1}{test}}~\texttt{#1}}

\usefonttheme[onlymath]{serif}

\titlebackground*{assets/background}

\newcommand{\hrefcol}[2]{\textcolor{cyan}{\href{#1}{#2}}}

\title{商业决策平台的智能数据管理系统}
\subtitle{数据结构与算法个人大作业}
% \course{Master's Degree in Computer Science}
\author{isKage}
% \IDnumber{1234567}
\date{2025年5月28日}

\begin{document}
\maketitle

\begin{frame}

% This template is a based on \hrefcol{https://github.com/TOB-KNPOB/Beamer-LaTeX-Themes}{Beamer-LaTeX-Themes} and its modified by ARCW
本项目基于 \texttt{Python} 开发,为更好的展示效果,采用 \texttt{Django} 网页框架,开发了一款“数据驱动商业决策平台”。
各个基础数据结构和算法均从零实现,本项目目标是尽可能的优化时间复杂度,并设计数据存储方式以适配更广泛的开发情景。

\vspace{\baselineskip}

项目存储在 Github ,链接 \hrefcol{https://github.com/isKage/bidms-dsa-pj}{https://github.com/isKage/bidms-dsa-pj} 。

% This template is released under \hrefcol{https://creativecommons.org/licenses/by-nc/4.0/legalcode}{Creative Commons CC BY 4.0} license

\end{frame}

\section{Introduction}

\begin{frame}{项目背景}

你是一家新兴的“数据驱动商业决策平台”的员工,公司要求你帮助设计其底层数据处理模块。你需要设计一个系统支撑平台的智能化运行,该系统存储的数据包括:

\begin{itemize}
\item 营销任务:每一条任务存储任务名称、紧急度、影响力;
\item 一个表示客户之间关系的加权有向图:图中的每个点表示一个客户并存储客户名称;
\item 所有商品的数据:每一条数据存储商品名称、价格、热度。
\end{itemize}
\end{frame}

\begin{frame}{文件结构}
\begin{itemize}
\item 本项目基于 \texttt{Python} 开发,为更好的展示效果,采用 \texttt{Django} 网页框架,开发了一款“数据驱动商业决策平台”网页雏形
\item 其中核心功能实现的代码见 \texttt{core/} 库
\item 核心功能具体实现 \texttt{services/}
\item 本项目实现的各类数据结构 \texttt{data\_structures/}
\end{itemize}
\end{frame}

% \begin{equation*}
% \mathrm{i}\,\hslash\frac{\partial}{\partial t} \Psi(\mathbf{r},t) =
% -\frac{\hslash^2}{2\,m}\nabla^2\Psi(\mathbf{r},t)
% + V(\mathbf{r})\Psi(\mathbf{r},t)
% \end{equation*}

\section{Task}

\begin{frame}{任务管理}
简单来说,对于一个任务,我们需要维护至少 3 个字段 \texttt{name, urgency, impact} ,并实现高效的增删改查功能,以及查看优先级最高的前 \texttt{k} 个任务。其中,一个任务的优先级定义为其紧急度和影响力的乘积。
\begin{itemize}
\item 本项目提出对每一个任务,都多维护一个字段 \texttt{uid} 表示该任务的唯一标识符。且 \texttt{uid} 满足主键的特征,即不可重复,且具有自增属性。
\item 本地存储:于是,我们可以用一个简单的表格存储任务列表:\texttt{uid, name, urgency, impact}
\item 内存存储:对于这样的一个任务列表,我们采用\textbf{优先级队列}的方式读入这些数据。为了能在 $O(1)$ 内查看某节点,故该优先级队列增加了定位器。
\end{itemize}
\end{frame}

\begin{frame}{功能实现}
我们定义一个类 \texttt{TaskService} 用来维护整个任务列表,并实现功能:
\begin{itemize}
\item 增:使用优先级队列的插入;
\item 删:使用 \texttt{uid} 快速定位后删除;
\item 改:改 name 则只需修改值;修改键,则采用先删后加的方式;
\item 查:使用 \texttt{uid} 快速查找。
\item \texttt{top k} :方法一是不断的从堆顶出堆;方法二是建立一个 \texttt{k-heap} ,然后和剩下元素进行比较,不断进行出堆入堆操作。
\end{itemize}
\end{frame}

\begin{frame}{算法分析}
首先声明,动态数组的摊销不再特地说明。因为是基于堆实现的优先级队列,所以对于 $n$ 个任务的情况,增(插入)和删除的时间复杂度为 $O(\log n)$ 。
而改值,使用 \texttt{uid} 进行 $O(1)$ 的查找,对于改键,则需要 $O(2\log n) = O(\log n)$ 。查找在基于 \texttt{uid} 映射的方法只需 $O(1)$ 。

top-k :
\begin{itemize}
\item 方法一,每一次堆顶元素(最小),复杂度为 $O(\log n)$ ,总共出堆 $k$ 次,故复杂度为 $O(k\log n)$ 。
\item - 方法二,先将 $k$ 个元素建堆 ,复杂度为 $O(k)$ 。然后将剩下的 $n-k$ 个元素逐个和堆顶元素比较,如果更大,则替换掉栈顶元素,同时进行向下冒泡 downheap 使得满足 heap-order 性质,这一步的复杂度为 $O((n-k)\cdot \log k)$ 。最后再逐个出堆,总共的时间复杂度为 $O(k + (n-k)\cdot \log k + k\log k) \sim O(n\cdot \log k)$ 。若不采用自底向上建堆,则为 $O(k\log k + (n-k)\log k + k \log k) \sim O(n\log k)$ 。
\end{itemize}

所以实现时,可以选取一个临界点,实现复杂度为 $\min \{ O(k\log n),\ O(n\log k) \}$ 。
\end{frame}


\section{Task Plus}

\begin{frame}{任务管理拓展}
\begin{itemize}
\item 简单来说,无需修改原数据存储方式,只需增加一个图数据存储任务直接的联系即可。
我们采用有向图,每个节点存储任务的 uid ,用 A 指向 B 表示 B 是 A 的前置。
那么这个图的要求就是,不能存在有向循环/有向环,即图为 DAG (有向非循环图)。
\item 图采用邻接映射实现,所以可以采用 json 格式文件存储图结构,而每个图的节点只存储 uid ,其他数据再之前的优先级队列中已经存储完成。
\end{itemize}
\end{frame}

\begin{frame}{功能实现}
\begin{itemize}
\item 首先,继承之前实现的 TaskService 类,并在初始化时先调用父类初始化方法。
\item 此时的增删改查功能,在基于父类已经实现的基础上,添加图的操作。
\item 对于有向无环图的限制判断,我们采用\textbf{拓扑排序算法}
\item 添加边 uv ,当拓扑排序包含了所有点时,说明不含有向循环,否则不允许增加边 uv 。
\end{itemize}
\end{frame}


\begin{frame}{算法分析}
\begin{itemize}
\item 插入/删除边对于图来说均为 $O(1)$ ,对于点的插入为 $O(1)$ ,而点的删除需要删除所有邻边,故为 $O(\deg v)$ 。点的查找,使用定位器只需 $O(1)$ ,而这里边的查找是基于点,故仍然为 $O(1)$ 。
\item 而拓扑排序的时间复杂度为 $O(n+m)$ 其中 $n$ 为点的个数(即任务数),$m$ 为边的个数(即任务关系数)。注意到,有无环是整个图的性质,一般而言最坏的情况下均要遍历整幅图。
\item 除了拓扑排序,也可以使用遍历(BFS or DFS),例如对于点 u 和 v ,我们想判断 u -> v 相连后会不会出现环。可以检查此时是否有一条从 v 到 u 的路径,即从 u 开始向上遍历,如果上游存在 v 则相连会成环。注意到,遍历的复杂度也是 $O(n+m)$ 。
\end{itemize}
\end{frame}


\section{Client}


\begin{frame}{客户管理}
\begin{itemize}
\item 简单来说, 需要维护客户 `name` 字段和客户之间关系影响 `weight` 字段。同样,我们采用 uid 代表每一个客户,允许客户重名。使用有向图代表关系,A 指向 B 代表 A 对 B 有影响,不妨设客户之间只能单向影响,即不允许两点之间有 2 条边,即简单图(但允许成环)。
\item 与 Task Plus 的存储方式相同,json 文件存储图结构,图的每个点存储 uid 图的边存储影响力。未存储客户的详细信息,使用 csv 文件存储 uid 和客户其他信息。
\end{itemize}
\end{frame}


\begin{frame}{功能实现}
\begin{itemize}
\item 相比于之前的 task ,客户管理功能只需完成存储信息和关系即可,没有比较和优先级的需求,直接使用图即可完成。
\item 注意,uid 到客户信息的映射,也可以采用存储 uid: offset 其中 offset 为客户 uid 数据在磁盘中的偏移量,这样就可以实现超大量数据的快速搜索,而无需读入内存。这一想法的实现,可见后面实现 product 的磁盘管理。
\item 而要查找被客户影响的其他所有客户,可以采用广度优先搜索 BFS 遍历全图,并计算沿路径的权重和作为其影响力。
\end{itemize}
\end{frame}


\begin{frame}{算法分析}
\begin{itemize}
\item 点和边的增删改查与 Task Plus 一致,这里不赘述。对于广度优先搜索,复杂度为 $O(n+m)$ ,其中 $n$ 为客户数, $m$ 为关系数。
\item 【Tip】之前的 Task 关系和这里的 Client 关系图均采用邻接表/映射实现,并存储在 json 文件中。如此加载数据,重新构建图十分方便,但也浪费了许多空间。可以考虑存储有向边集合到文件中,每次加载读取时,从边集加载重构图结构。这样并不会消耗过多时间,只要在项目开始运行时就先加载数据,项目结束时才存储数据到文件,如此重构图的过程只进行了一次。
\end{itemize}
\end{frame}


\section{Client Plus}


\begin{frame}{客户管理拓展}
\begin{itemize}
\item 拓展:当图中两个客户间的路径过长或路径影响力过小时,这两个客户间的影响可以忽略不计。在此基础上,完善这一部分功能。
\item 总影响力有多种定义:所能到达的所有路径的权重和;所有最大路路长的平均
\end{itemize}
\end{frame}


\begin{frame}{功能实现}
下面我们实现第二种定义——所有最大路路长的平均。
\begin{itemize}
\item 修改 Dijkstra 算法,使得当此时的权重和超过某个阈值,则停止向下探索。即实现了“路径过长或路径影响力过小时,这两个客户间的影响可以忽略不计”
\item 我们也可以基于统计推断的思想,先利用 Floyd Warshall 算法计算各点之间的最短/大路,然后进行统计分析
\end{itemize}

统计分析:

设 $X_{ij}$ 代表客户 $i$ 对客户 $j$ 的最大影响力大小。假设 $X_{ij} \sim N(\mu_i,\ \sigma_i^2)$ ,且 $X_{ij}$ 相互独立对于 $\forall\ j = 1,\ 2,\ \cdots,\ n$ 。则有:

\begin{equation*}
\hat{\mu}_{i,\ unb} = \frac{1}{n}\sum_{j=1}^n X_{ij} = \bar{X_i}
\end{equation*}

\begin{equation*}
\hat{\sigma^2}_{i,\ unb} = \frac{1}{n-1}\sum_{j=1}^n (X_{ij} - \bar{X_i})^2 = S^2
\end{equation*}
\end{frame}


\begin{frame}{功能实现}
\begin{equation*}
\frac{X_{ij} - \mu_i}{\sigma_i} \sim N(0,\ 1)
\end{equation*}

\begin{equation*}
P\left( z_{\alpha/2} < \frac{X_{ij} - \mu_i}{\sigma_i} < z_{1-\alpha/2} \right) = 1 - \alpha
\end{equation*}

我们定义 $X_{ij} \leq \mu_i + z_{\alpha/2}\cdot \sigma_i$ 为异常小的,可以忽略不计。于是根据已有的数据 $X_{ij}\quad \forall j = 1,\ 2,\ \cdots,\ n$ 可以得到被忽略的数据 $X^{ign}_{ij}$(基于对 $\mu_i$ 和 $\sigma$ 的估计):

\begin{equation*}
X^{ign}_{ij} \leq \left(\frac{1}{n}\sum_{j=1}^n X_{ij}\right) + z_{\alpha/2}\cdot  \left( \frac{1}{n-1}\sum_{j=1}^n (X_{ij} - \bar{X_i})^2\right)
\end{equation*}
\end{frame}

\begin{frame}{算法分析}
\begin{itemize}
\item 方法一:Dijkstra 算法使用优先级队列堆实现的复杂度为 $O(n+m)\log n$ 而我们需要对 $n$ 个客户使用,故复杂度为 $O(n(n+m)\log n)$ 。
\item 方法二:Floyd Warshall 算法复杂度为 $O(n^3)$ 。
\item 当图稀疏时,$O(n(n+m)\log n) = O(n^2 \log n)$ 肯定是优于 $O(n^3)$ 。但当图稠密时, $O(n(n+m)\log n) = O(n \cdot n^2 \log n) = O(n^3\log n)$ 是劣于 $O(n^3)$ 。
\end{itemize}
\end{frame}


\section{Product}


\begin{frame}{商品管理}
\begin{itemize}
\item 简单来说,我们需要维护一个商品的 `name, price, popularity` 字段,同样我们维护一个 `uid` 作为唯一标识符,实现增删改查,并返回任意 price 区间的数据,按照 popularity 排序。同时实现字符串匹配问题。
\item 类似 Task 基础功能,使用一个 csv 文件存储每个商品信息:uid,name,price,popularity
\item 内存数据:读入数据,存储在一个有序映射 AVL 树(二叉搜索树),这里的 AVL 树的键存一个特殊的对象 `ProductKey` ,而节点值为一个哈希表,作为一个桶,存储多个 `Product` 对象。
\end{itemize}
\end{frame}

\begin{frame}{功能实现}
\begin{itemize}
\item 考虑一种情况:我们需要按照 price 范围返回对象,所以 AVL 树的键应该是 price 相关的数据。但二叉搜索树如何处理键相同的问题?我们使用的 `AVLTreeMap` 类是一个有序映射,映射则不允许键相同。所以一个解决方法就是用桶 Bucket 的思想。
\item 每个搜索树节点上存储的不是一个 `Product` 类,而是一个哈希表,哈希表中存储着多个商品,形如 uid: Product 。所以我们建立的 `Product` 类处理维护 `uid,name,price,popularity` 还需要维护一个 `self.bucket` 用来表示当前商品属于哪个桶。
\end{itemize}
\end{frame}

\begin{frame}{功能实现}
\begin{itemize}
\item 除此之外,直接将 \texttt{price} 作为键也是一个不好的选择。其一,\texttt{price} 为浮点数,浮点数的 \texttt{==} 运算精度有限,简单来说就是 \texttt{float} 类型\textbf{全序性质}不佳。其二,注意到除了 \texttt{price} ,热度 \texttt{popularity} 也是排序的标准之一,简单来说,这是一个\textbf{字典序排序}。其三,目前只有 2 个属性可用来排序,未来面对更复杂的情况,则无法使用,代码可拓展性差。
\item 所以,我们可以定义一个\textbf{键类}例如本例的 \texttt{Product.ProductKey} 类,使其满足全序性质(定义 \texttt{\_\_eq\_\_}, \texttt{\_\_lt\_\_}),同时定义哈希函数(\texttt{\_\_hash\_\_}),使其具有充当键的完备性和可拓展性。
\end{itemize}
\end{frame}


\begin{frame}{功能实现}
\begin{itemize}
\item 插入:获取 uid 并实例化一个 `Product` 对象,查看是否 AVL 树的键中是否已经有 `ProductKey` 。没有则新建桶,并插入;有则直接插入。
\item 删除:根据 uid 和 `self.uid\_map` 快速找到 Product 类,并获取 bucket ,从桶中删去,检查桶是否为空,空则从 AVL 树中删去键为此的节点。
\item 更新:若只更新与键无关的值,只需找到 Product 类修改即可,对 AVL 不操作;否则先删除后插入。
\item 查看:根据 uid 查看。
\item price 范围:找到 `start` 位置,然后中序遍历,逐个返回即可,直到 `end`。
\end{itemize}
\end{frame}

\begin{frame}{模式匹配}
Knuth-Morris-Pratt 算法:
\begin{itemize}
\item 方法一:逐个比较商品 name 和 pattern 。
\item 方法二:对于目标搜索 pattern ,先将所有商品的 name 使用特殊字符(例如 "*")拼接成一个长字符串,然后进行匹配 pattern ,返回所有匹配成功的位置。
\item 【思考】当 name 列表过大,是否会浪费过多空间。是否可以考虑使用霍夫曼编码,压缩 text 文本和 pattern 字符串,对二进制数按位计算,一次节省空间。例如,ASCII 码一个字符占据 1 个字节,但在霍夫曼编码的情况下能压缩。可以尝试构造一个长数组,分配空间为 `len * 1 byte` (C 语言中最小寻址为 1 字节),那么 1 字节可以存储不止 1 个字符。
\end{itemize}
\end{frame}


\begin{frame}{算法分析}
\begin{itemize}
\item 设商品数为 $n$ ,则 AVL 树高度大约为 $h = O(\log n)$ 。搜索的复杂度为 $O(\log n)$ 。插入和删除也为 $O(\log n)$ 。更新操作最坏也是插入和删除的组合,仍然为对数时间。查找范围为 $O(s + \log n)$ 其中 $s$ 为查找范围中含有的商品数。
\item 模式匹配:设商品名大约均为 $m$ 长,匹配的字符串长为 $p \leq m$ ,总共 $n$ 个商品。如果采用方法一,长为 m 的匹配复杂度为 $O(m + p)$,循环 n 次,最终复杂度为 $O(n(m + p))$ ,方法二先遍历 $O(n)$ ,然后拼接字符串,复杂度为 $O(n)$ ,然后对整体匹配,复杂度为 $O(nm + p)$ ,最后索引会原来的 uid $O(s)$ ,其中 $s \sim n$ 为匹配到的位置最终复杂度为 $O(n +nm + p)$ 。二者的差:
\end{itemize}
\begin{equation*}
n(m+p) - (n + n + nm + p + s) = np-2n-p-s = (n-1)(p-1) - n - s - 1 \geq 0
\end{equation*}

一般当 $p \geq 4$ 时,方法二更优,因为 s 一般与 n 同一量级。不过两种方法并无太大差距,而且方法二明显浪费空间。

\end{frame}


\section{Product Plus}


\begin{frame}{商品管理拓展}
\begin{itemize}
\item 实际上,由于该平台上的商品数目过多,所有的商品数据存储在磁盘上,且无法被全部读取到内存后进行处理。
\item 计算机系统提供多种数据存储方式:基于芯片的主存(内存) \& 基于磁盘的辅存。主存存储正在运行的程序和数据,断电后数据丢失,存储数据比辅存少,但计算速度极快。辅存基于磁盘,读取依靠机械运动,可存储数据巨大,但读写速度非常慢(比内存慢 10 万倍)。(为提高效率,磁盘一般按块 block 读入,而内存会实现分页 page 操作,实现与磁盘块的对齐,当内存访问不到某个对象,触发缺页异常,会将磁盘读取的块数据按照页读入内存)。
\item 所以在考虑这个问题时,我们需要将磁盘读取的时间考虑在内。
\end{itemize}
\end{frame}

\begin{frame}{B 树与 B+ 树}
\textbf{B 树} 是一种满足以下条件的树结构:
\begin{itemize}
  \item \textbf{多路分支}:每个节点最多有 $t$ 个子节点($t$ 阶)。
  \item \textbf{平衡性}:所有叶子节点位于同一层。
  \item \textbf{子节点限制}:
  \begin{itemize}
    \item 根节点至少有 2 个子节点(除非它是叶子)。
    \item 内部节点至少有 $\left\lceil t/2 \right\rceil$ 个子节点。
    \item 每个节点最多有 $t$ 个子节点。
  \end{itemize}
  \item \textbf{键限制}:
  \begin{itemize}
    \item 每个节点最多存储 $t - 1$ 个键。
    \item 非根节点至少有 $\left\lceil t/2 \right\rceil - 1$ 个键。
  \end{itemize}
  \item \textbf{键值分布}:每个节点中键值有序排列,键 $k_1$ 与 $k_2$ 之间的子树满足 $k_1 \leq k \leq k_2$。
\end{itemize}
\end{frame}

\begin{frame}{B+ 树特点}
\textbf{B+ 树} 是 B 树的变种,具备以下特点:
\begin{itemize}
  \item \textbf{内部节点}:不存储实际数据,仅作为索引使用。
  \item \textbf{叶子节点}:才存储完整的数据记录。
  \item \textbf{链式连接}:所有叶子节点通过指针串联,便于范围查询。
\end{itemize}

本项目的 b\_plus\_tree.py 文件,实现了 B+ 树 BPlusTree 类以及结合具体 Product 商品数据的管理(增删改查+范围搜索),
采用的是一次性将所有节点索引读入内存,叶子节点存储了具体数据的磁盘索引,并借用 shelve 高级库进行磁盘管理,下面介绍从零实现的 B 树以更好的理解。
\end{frame}

\begin{frame}{数据读取与磁盘管理}
\begin{itemize}
  \item 以 B 树为例,结合本项目的商品数据,介绍如何处理超大量数据。
  \item 数据存储在二进制文件 \texttt{btree.db} 中,保存的是 B 树的节点对象。
  \item 每个节点包含以下信息:
  \begin{itemize}
    \item \texttt{is\_leaf}:是否为叶子节点;
    \item \texttt{key}:键值,如商品价格 \texttt{price};
    \item \texttt{children}:子节点在磁盘中的偏移量(offset);
    \item \texttt{offset}:当前节点自身的磁盘位置(通常只存根节点);
    \item \texttt{n}:当前节点包含的键数量。
  \end{itemize}
\end{itemize}
\end{frame}

\begin{frame}{节点磁盘存储结构}
\begin{itemize}
  \item 每个节点在磁盘上分配固定长度的 bit 空间,包含如下字段:
  \begin{itemize}
    \item \texttt{is\_leaf}:1 bit,标记是否为叶子节点;
    \item \texttt{keys}:每个键分配 \texttt{KEY\_SIZE = 8} 字节(64 bit),共预留 \texttt{MAX\_KEYS = 2T - 1} 个键;
    \item \texttt{children}:每个指针占用 \texttt{OFFSET\_SIZE = 8} 字节,总计 \texttt{CHILDREN\_SIZE = MAX\_KEYS + 1 = 2T} 个位置;
    \item \texttt{n}:记录当前键的数量,占用 8 字节;
    \item \texttt{offset}:记录当前节点在磁盘中的位置,也预留 8 字节。
  \end{itemize}
  \item \textbf{说明}:\texttt{offset} 表示该节点位于磁盘的第 \texttt{offset // page} 页的第 \texttt{offset \% page} 个字节位置。可以类比为磁头旋转定位的位置偏移。
\end{itemize}
\end{frame}

\begin{frame}{B 树的搜索操作}
\begin{itemize}
  \item 搜索键值 $k$ 的过程:
  \begin{enumerate}
    \item 读取根节点:根据 \texttt{root\_offset} 从磁盘中加载根节点。
    \item 当前节点查找:在节点的 \texttt{keys} 中搜索 $k$,若找到直接返回。
    \item 若未命中,根据 $key_i < k < key_j$ 选择子节点 $children_i$ 对应的 offset。
    \item 从磁盘加载该 offset 节点并重复步骤 2。
    \item 若最终在叶节点未找到,返回 \texttt{None}。
  \end{enumerate}
  \item 插入/删除涉及结构变更,具体见源码:\texttt{b\_tree\_disk.py}
  \item \textbf{说明}:更多内容可以参考教材 \href{https://book.douban.com/subject/1152912/}{《算法导论》} 第五部分-高级数据结构-第 18 章-B树。
\end{itemize}
\end{frame}

\begin{frame}{B 树 vs B+ 树:数据存储策略}
\begin{itemize}
  \item B 树优点:
  \begin{itemize}
    \item 搜索路径上的节点按需加载;
    \item 节省内存,适合大数据环境。
  \end{itemize}
  \item B 树 \texttt{keys} 实际应存储完整数据(如 \texttt{Product} 对象),比较时应使用其 \texttt{ProductKey}。
  \item B+ 树差异:
  \begin{itemize}
    \item 内部节点仅存键值索引,不含具体数据;
    \item 叶子节点与 B 树类似,但多一个指向下一个叶子的 offset 指针;
    \item 更适合范围查询与顺序遍历。
  \end{itemize}
\end{itemize}
\end{frame}



\begin{frame}{B 树算法复杂度分析}
设 B 树包含 $n$ 个键,每个节点至多 $t$ 个子节点,高度 $h$ 满足:
\[
h \leq \log_t \frac{n+1}{2}
\]
结论如下(证明参见《算法导论》):
\begin{itemize}
  \item \textbf{搜索}:
  \begin{itemize}
    \item 磁盘访问:$O(h) = O(\log_t n)$,即每层读一次;
    \item CPU 内存访问:每个节点含最多 $t - 1$ 个键,故为 $O(t \log_t n)$。
  \end{itemize}
  \item \textbf{插入}:
  \begin{itemize}
    \item 与搜索相同;
    \item 若节点满则分裂,操作为 $O(1)$。
  \end{itemize}
  \item \textbf{删除}:
  \begin{itemize}
    \item 与插入相同;
    \item 若涉及合并/向下合并,复杂度仍为 $O(1)$。
  \end{itemize}
\end{itemize}
\end{frame}

\section{Conclusion}

\begin{frame}{项目总结}
本项目从零实现了基础数据结构,例如优先级队列、树、图、哈希表,以及一些基础算法,例如排序算法、图的各类算法、模式匹配等,并在此基础上拓展了 B 树和 B+ 树的知识,同时实现了基于磁盘管理的 B 树。
\end{frame}

\begin{frame}{优点}
\begin{itemize}
  \item \textbf{设计数据存储方式}:每一个对象,使用自增唯一的主键 \texttt{uid} 进行唯一标识,为未来复杂数据的开发提供了便利。
  \item \textbf{提出了基于比较的自定义“键”对象}:在基于比较的排序算法中,针对“键”的全序要求,本项目提出了自定义“键“类,从而可以自定义”比较“的方式,为未来处理复杂比较问题提供了便利。
  \item \textbf{处理了”键“相同的冲突}:许多场景允许键相同(例如价格作为键排序时),本项目提出了建立桶 Bucket 结构来存储多个相同键的数据。
  \item \textbf{实现了基于磁盘管理的 B 树}:B+ 树借用了高级包 \texttt{shelve} ,此包以类字典的方式管理文件读写;除此之外,本项目还完成了从零实现的基于磁盘管理的 B 树,数据以二进制文件的方式存储,完成了基本的增删改查和对应的永久性处理。
  \item \textbf{项目已实现可视化 UI 界面}:本项目基于 `Django` 实现了前端编写,完成了项目雏形设计。
\end{itemize}
\end{frame}


\begin{frame}{项目不足与改进方向}
\textbf{当前不足:}
\begin{itemize}
  \item 为与前端兼容,实现了一些与算法和结构本身无关的方法,封装程度不足。
  \item 问题二用户影响部分,对影响的定义比较简单。
  \item B 树磁盘管理部分使用了一些不必要的文件读写,效率有所影响。
  \item B+ 树实现依赖 \texttt{shelve},细节控制有限。
\end{itemize}

\vspace{0.5em}
\textbf{未来改进方向:}
\begin{itemize}
  \item 有关网络重要性可以采用 PageRank 算法,前缀查找可以采用字典树结构。
  \item 磁盘按“块”方式批量读入,提高 I/O 效率;
  \item 探索 KPM 算法是否可以与 Huffman 编码联合使用,提高内存利用率(待思考)。
  \item 使用 uid 映射到具体数据时,也可以采用 B 树磁盘管理的思想,由 uid 映射到该数据对应的偏移量,从而可以应对一条十分巨大的数据记录。
\end{itemize}
\end{frame}

\backmatter
\end{document}
